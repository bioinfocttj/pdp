\documentclass[12pt,a4paper]{report}
\usepackage[utf8x]{inputenc}
\usepackage[T1]{fontenc}
\usepackage{lmodern}
\usepackage{ucs}
\usepackage{amsmath}
\usepackage{amsfonts}
\usepackage{amssymb}
%\usepackage{fullpage}
\usepackage[french]{babel}
\usepackage{xcolor}
\usepackage[pdftex]{graphicx}
\usepackage{titlesec}
\usepackage{cite}
\usepackage{pdfpages}
\usepackage{listings}
\usepackage{url}
\usepackage{rotating}
\usepackage[top=2cm,bottom=2cm,left=2.5cm]{geometry}

%%%%%%%%%%%%%%%%%encadrementdes chapitres%%%%%%%%%%%%%%%%%%%%%%%
\titleformat{\chapter} % commande de sectionnement affectée
[frame] % une des formes prédéfinies
{\itshape} % format appliqué au titre dans son ensemble
{\filright\small\enspace Chapitre \thechapter\enspace} % format du « n° » du titre
{8pt} % distance (horiz. ou vert.) entre le n° et le texte du titre
{\Large\bfseries\filcenter} % format appliqué au texte du titre
%%%%%%%%%%%%%%%%%%%%%%%%%%%%%%%%%%%%%%%%%%%%%%%%%%%%%%%%%%%%%%%%
\newcommand{\HRule}{\rule{\linewidth}{0.5mm}}
\title{Rapport Projet de programmation}
%\author{Charlotte \textsc{Herice}\\ Typhaine \textsc{Paysan-Lafosse}\\ Thomas \textsc{Faux}\\ Joris \textsx{Sansen}}
% Commenté car provoque une erreur lors de la compilation
\begin{document}
%\maketitle
\input{./titleRapport.tex}
\newcommand{\cme}{cryo-MET}
\newcommand{\java}{Java~{\tiny \texttrademark}}
\newcommand{\js}{JavaScript}
\newcommand{\imj}{ImageJ}

\begin{abstract}
Le contexte de ce projet est l'étude de morphologies de virus ou de patchs de virus observés au microscope électronique (ME) dans le cadre d'une prestation de services avec une société pharmaceutique. Le projet consiste en le développement d'un utilitaire qui sélectionne automatiquement les particules virales et les protéines membranaires dans les images de \cme. Cet utilitaire se présente sous la forme d'un plugin \imj, écrit en  langage \java.
Il aborde notamment la création de l'interface utilisateur, les méthodes de piquages choisies et implémentées, ainsi que l'organisation générale du programme. 
\end{abstract}

\tableofcontents
\chapter*{Introduction}

\addcontentsline{toc}{chapter}{Introduction}
Notre projet s'est déroulé au sein du laboratoire de Chimie et Biologie des Membranes et Nanoobjets de Bordeaux (CBMN~\cite{cbmn:url}).
Il s'agit d'un laboratoire de recherche public composé de douze équipes de recherche dont l'équipe \emph{Architecture des Complexes Membranaires et processus cellulaires} (ACMPC). %
Cette équipe, dirigée par O.\textsc{Lambert}, s'intéresse à l'architecture de complexes membranaires sur des structures de type protéine-liposome. C'est dans ce cadre d'étude que les chercheurs travaillent avec un \emph{cryo-microscope électronique à transmission} (\cme) afin d'obtenir des micrographes des structures protéiques puis de les analyser. Par ailleurs, cette équipe diversifie ses activités en faisant de l'imagerie avec des virus fournis par une entreprise pharmaceutique.

\noindent
Cependant la nouvelle génération de \cme ~permet une collecte de données à haut débit, avantage non négligeable mais qui pose le problème du temps de traitement des données collectées. %

\paragraph*{}
Dans le cadre de leurs recherches et pour la partie qui nous intéresse, l'analyse concerne le traitement des images récupérées du \cme\ et plus particulièrement au \emph{picking}, c'est-à-dire au piquage des particules présentes sur les micrographes obtenus. %
Notre objectif était l'implémentation d'une plateforme contenant plusieurs méthodes de piquage automatisées.
Celle-ci est implémentée en \java ~\footnote{\java\ est un langage orienté-objet développé par Oracle~\cite{java:url}} sous la forme d'un plugin \imj~\cite{imagej:url}: elle propose à l'utilisateur des algorithmes de picking pré-installés ou d'en ajouter de nouveaux.

\paragraph*{}
Dans la suite de ce rapport, nous développerons plusieurs points. Tout d'abord une partie Analyse dans laquelle nous remettrons notre projet dans son contexte et analyserons les besoins liés à celui-ci. Ensuite, une partie Conception dans laquelle nous expliquerons l'organisation de notre code. Enfin, la partie Réalisation contiendra les solutions apportées au sujet.

\chapter{Analyse}

\section{Contexte}

\subsection{Présentation du CBMN}

Le laboratoire, créé en janvier 2007, est à l'interface entre la Biologie, la Chimie et la Physique. Il a pour tutelle les départements de Chimie et des Sciences de la Vie du Centre National de la Recherche Scientifique (CNRS) et les Unités de Formation et de Recherche (UFR) de Chimie et de Biologie de l'Université Bordeaux I, II et l'ENITAB. \\
La mission du CBMN est d'apporter une connaissance fondamentale de phénomènes biologiques complexes en les analysant à plusieurs échelles, allant de la molécule à la cellule et à l'organisme. \\
L'équipe ACMPC s'appuie sur deux méthodes d'imagerie : la cryotomographie  électronique et la microscopie à fluorescence. Comme nous l'avons expliqué précédemment, nous allons travailler sur des images issues de cryomicroscopie.  

\subsection{Objectifs}

\noindent
Notre objectif était de créer et d'implémenter une plateforme mettant à disposition plusieurs méthodes de piquage automatisées des particules sur les micrographies. Les échantillons qui nous ont servi de tests étaient de deux types :%
\begin{itemize}
\item des micrographies d'échantillons de virus, de forme circulaire, qu'il fallait sélectionner afin de pouvoir déterminer leurs nombre et tailles,
\item des micrographies de protéines membranaires, de forme pyramidale, qu'il fallait aussi sélectionner.\\
\end{itemize}
\noindent
Il était imposé que cette interface soit implémentée sous la forme d'un plugin ImageJ. Elle propose à l'utilisateur de se servir des algorithmes de piquage pré-installés. \\%

\noindent
Le logiciel ImageJ a été choisi car c'est un logiciel de traitement et d'analyse d'images développé en \java ~par le Nation Institute of Health (NIH).%

\noindent
Ce logiciel est "open-source"~\footnote{Son code source est dans le domaine public et la plupart des plugins sont sous licence GNU-GPL}, multi-plateforme et bien connu de la communauté scientifique car initialement conçu pour des applications biomédicales. Il s'est peu à peu démocratisé dans d'autres domaines pour sa facilité d'utilisation et les possibilités de développement qu'il offre.%

\noindent
En effet, il est possible de développer soi-même et assez facilement des plugins, que ce soit en \java\ ou en \js ~\footnote{\js ~est un langage de programmation de script orienté objet à prototype~\cite{javascript:url}}. \\

La figure ci-après donne un aperçu visuel de ce que nous voudrions obtenir au final. Nous pouvons y retrouver l'interface, l'image étudiée avec les particules sélectionnées, la pile d'images individuelles ainsi que le tableau de résultats. 

\begin{figure}[!ht] 
\begin{center}
\includegraphics[width=1\textwidth]{Visuel.png}
\caption{Rendu visuel attendu du plugin de picking}
\label{visuel}
\end{center}
\end{figure}

\section{\'Etat de l'existant}

\subsection{Détection à l'aide de références}

Cette technique est utilisée pour trouver des particules dans une image en la comparant à un modèle. Celui-ci peut \^etre obtenu soit à partir d'une structure tridimensionnelle connue, soit par sélection d'une particule servant d'exemple dans les micrographies. L'algorithme détermine la meilleure correspondance entre la cible et le modèle pour pouvoir le localiser dans l'image.%

\subsection{Piquage de particules sans référence}

\subsubsection{Détection des bords et transformée de Hough}

%$\textcolor{red}{(ref:Automatic Particle Detection Through Efficient Hough Transforms
%by Yuanxin Zhu, Bridget Carragher, Fabrice Mouche, and Clinton S.Potter
%IEEE TRANSACTIONS ON MEDICAL IMAGING,VOL.22,N0.9,September 2003)}$\\


%$\textcolor{red}{(ref:http://en.wikipedia.org/wiki/Hough_transform)}$\\
\paragraph*{}
Les difficultés majeures rencontrées dans les techniques basées sur la détection des contours sont dues à la complexité de détecter les bords des particules lorsqu'il y a un bruit de fond important sur les images de MET.%

\noindent
Cette technique est basée sur la détection des contours ainsi que sur l'application de la transformée de Hough~\cite{PdetectEHT:article}. Cette méthode permet de détecter la présence de formes comme des lignes, des cercles ou encore des ellipses.%

\noindent
Dans la transformée de Hough\cite{HT:url} appliquée à la détection de lignes, pour chaque pixel allumé de l'image on trace toutes les lignes possibles, on obtient alors une sinusoïde unique appelée espace de Hough. Si les courbes associées à deux points se coupent, l'endroit où elles se coupent dans l'espace de Hough correspond aux paramètres d'une droite qui relie ces deux points (ordonnée à l'origine et pente).%

\paragraph*{}
La détection de formes qui nous intéressent s’apparentant à des cercles ou des axes circulaires se fait en détectant le centre de celles-ci.
Pour chaque pixel allumé, la fonction trace un cercle de rayon donné. Si la forme a le même rayon que le cercle tracé, tous les cercles se recoupent au centre de la forme, on constate ainsi une amplification de la valeur du pixel central.

\paragraph*{}
D'autre part, pour détecter des particules de formes irrégulières, l'approche décrite précédemment peut également \^etre utilisée mais la transformée de Hough devra alors \^etre remplacée par la transformée de Hough Généralisée~\cite{GHT:url}.
Cette nouvelle méthode repose sur la modification de la transformée de Hough en utilisant le principe de l'identification à partir d'un modèle de référence.
%$\textcolor{red}{(ref:http://en.wikipedia.org/wiki/Generalised_Hough_transform)}$\\

\subsubsection{DoGLFC et classement par affinité}

%$\textcolor{red}{(ref: Reference-free particle selection enhanced with semi-supervised machine learning for cryo-electron microscopy
%by Robert Langlois, Jesper Pallesen, Joachim Frank
%Journal of Structural Biology 175, (2011)353-361)}$
\paragraph*{}
Cette technique est basée sur l'utilisation de l'algorithme DoGLFC\footnote{Difference Of Gaussian Local Fast Correlation} complété par l'algorithme de classement par affinité~\cite{DoGAff:article}.

\paragraph*{}
Le DoGLFC est basé sur l'algorithme DoG Picker du Scripps Institute\cite{Scripps:url}, une méthode rapide qui permet la segmentation de particules. Après l'application de l'algorithme de Différence de Gaussiennes (DoG), on obtient une cartographie de points similaire à celle de la méthode utilisant un modèle de référence.%\blue {\Large ?!} \black %

\noindent
Cet algorithme requiert un paramètre ajustable unique ou un jeu de paramètres basé sur le rayon de la particule ou un ordre de grandeur du rayon. L'exécution de cet algorithme renvoie une liste de trois paramètres décrivant la localisation de la particule (les coordonnées \emph{x} et \emph{y}) et la hauteur du pic.%

\noindent
L'algorithme DoGLFC sélectionne les particules (ou objets) potentielles de taille déterminée, ceci a un pouvoir discriminatif moindre par rapport à une technique basée sur un modèle de référence.

\paragraph*{}
Pour améliorer le rendement lors du piquage des particules, un nouvel algorithme semi-supervisé, le classement par affinité, peut \^etre appliqué.

\noindent
L’algorithme a besoin de trois paramètres d'entrée: un jeu d'images, la taille minimale de l'image après réduction et deux références indiquant quelle fen\^etre doit \^etre utilisée comme référence positive ou négative.

\noindent
Lorsque l'algorithme a fini de se dérouler, on obtient une classement pour chaque fen\^etre où le maximum correspond à la particule ciblée.

\paragraph*{}
L'utilisation de DoGLFC complété par le classement par affinité permet d'extraire
rapidement les particules de l'image et d'éliminer avec précision les particules correspondant à de la contamination ou du bruit de fond.

\subsection{Perceptron}

\paragraph*{}
Un perceptron est une sorte de réseau neuronal artificiel. Dans son état le plus simple, il représente un système de classification binaire/linéaire.%

\noindent
Ce programme a la particularité d'être capable d'apprendre des concepts, ce qui signifie qu'il peut apprendre afin de répondre par vrai ou faux à des données qui lui sont soumises, gr\^ace à la présentation répétée de plusieurs exemples d'étude.
Il a déjà été testé sur des images binaires pour la détection de formes ou de contours mais pas sur des images en niveaux de gris ou sur des problèmes de reconnaissance de modèles visant à sélectionner des particules. Le réseau neuronal n'a pas non plus été exploité comme un outil de sélection automatique de particules mais plusieurs recherches ont conclu qu'il pourrait \^etre utilisé pour l'élimination de faux-positifs.~\cite{Perceptron:article}.%

\section{Analyse des besoins}

\noindent
Ils sont de deux types: fonctionnels et non fonctionnels, et diffèrent entre l'interface utilisateur et les algorithmes de piquage.

\subsection{Besoins fonctionnels}

\subsubsection{Interface}

\noindent
Au lancement, les images sont préalablement chargées sur \imj , l'utilisateur a le choix entre plusieurs algorithmes de piquage. %L'interface propose un mode de prévisualisation afin de vérifier le piquage sur une image avant de l'appliquer au stack entier.
Nous avons essayé de faire en sorte que l'affichage soit clair et succinct, constitué d'un menu déroulant, d'une liste de boutons et d'une interface propre à chaque algorithme.\\
Enfin, il est également possible d'implanter simplement de nouveaux algorithmes dans l'interface. % Simplement, ou pas ^^

\subsubsection{Les algorithmes}

Les algorithmes implantés doivent réaliser un piquage automatique des particules depuis les micrographies issues de \me. Le format de sortie est un tableau de résultats contenant un jeu de coordonnées \emph{x, y} associé à la position de l'image dans la pile d'images (\textit{stack} en anglais), ainsi qu'une nouvelle pile contenant les particules sélectionnées par l'algorithme si l'utilisateur le désire.

\subsection{Besoins non fonctionnels}

\subsubsection{Interface}

\noindent
L'implémentation de la plateforme a été réalisée en \java, nous utilisons les API\footnote{Application Programming Interface} graphiques de la bibliothèque \emph{Swing}. 
Il s'agit d'une interface graphique (GUI)\footnote{Graphic User Interface} faisant partie du paquetage Java Foundation Classes (JFC), inclus dans J2SE. Cela constitue une des principales évolutions apportées par Java2 par rapport aux versions antérieures. Elle offre la possibilité de créer des interfaces graphiques identiques quelque soit de système d'exploitation sous-jacent, au prix de performances moindres qu'en utilisant Abstract Window Toolkit (AWT) l'autre bibliothèque principale pour \java. \\
De plus, nous avons essayé de faire en sorte que le programme soit le plus simple d'utilisation possible afin de le rendre accessible à tous. C'est pourquoi nous avons tenté de faire un code clair, explicite et commenté afin de permettre aisément l'implémentation de nouveaux algorithmes.

\subsubsection{Les algorithmes}

\noindent
Les algorithmes ont été testés avec l'outil macro d' \imj ~puis implémentés en \java. Nous avons fait en sorte que le temps de déroulement de l'algorithme soit assez rapide afin de pouvoir gérer de grands jeux de données et qu'aucune image de traitement intermédiaire n'apparaisse à l'utilisateur (à moins qu'il ne le souhaite).\\

Dans l'optique de démocratiser l'utilisation de notre interface et afin de permettre l'extension de ce plugin avec d'autres algorithmes, le projet a été placé sous une licence GPL\footnote{General Public License~\cite{GPL:url}}.

\noindent
%Le projet devra être terminé vers mi-mai. %prière de nous mettre une bonne note !

\section{Scénario d'utilisation du plugin}

Nous prendrons l'exemple d'une image issue de \cme ~collectée dans l'équipe ACMPC sur laquelle se trouvent des protéines membranaires que nous voulons sélectionner (Figure \ref{prot}) ainsi que d'une image utilisée comme référence sur \imj ~, les \emph{blobs} (Figure \ref{blobs1}).

\begin{figure}[!ht]
\begin{center}
 \begin{minipage}{.450\linewidth}
  \includegraphics[width=0.92\textwidth]{blobs.png}  
  \caption{Image Blobs de réference}
  \label{blobs1}
 \end{minipage} \hfill
\begin{minipage}{.450\linewidth}
  \includegraphics[width=0.98\textwidth]{proteines.jpg}   
  \caption{Image de protéines membranaires observées au \me}
  \label{prot}
 \end{minipage} \hfill
%\caption{Exemple d'images formant le stack de particules (blobs et protéines)}
\end{center}
\end{figure}

%\begin{figure}[h] 
%\begin{center}
%\includegraphics[width=0.5\textwidth]{proteines.jpg}
%\caption{Image de protéines membranaires observées au \me}
%\label{prot}
%\end{center}
%\end{figure}

Nous avons tenté de nous mettre à la place d'un chercheur pour avoir une vision biologique de l'utilisation de notre plugin. Tout au long de ce rapport, vous pourrez donc suivre toutes les procédures qui nous ont permis de réaliser la sélection. 


\chapter{Conception}

\section{Interface graphique}

Comme précisé précédemment, notre programme se présente sous la forme d'un plugin \imj. Il a été réalisé grâce à la bibliothèque Swing de \java ~pour l'interface graphique. Celle-ci offre la possibilité de créer des interfaces graphiques identiques quelque soit le système d'exploitation sous-jacent.\\
Nous avons donc créé une série d'outils graphiques (boutons, listes, labels) pour l'interaction avec l'utilisateur. Dans une fenêtre, ces outils sont disposés en lignes ou en colonnes dans des boîtes (panels). Les boîtes peuvent s'imbriquer les unes dans les autres afin d'obtenir des structures plus complexes et un meilleur rendu visuel. \\
L'interface du plugin est simple, elle prend la forme d'une fenêtre composée d'un menu déroulant, dans lequel se trouvent les différents algorithmes, et de quatre boutons. \\

\subsection{Les panels}

Notre interface est composée de quatre panels différents (voir Figure \ref{panels}) :
\begin{itemize}

\item Le panel en haut de la fenêtre (\textbf{panel1}) contient la combobox pour le choix de l'algorithme. 
\item Le panel du milieu (\textbf{panel2}) est vide au lancement du plugin et son contenu s'affiche en fonction de l'algorithme de piquage choisi. Il contient 4 sous-panels :
\begin{itemize}
\item Le premier sous-panel (\textbf{infoPanel}) contient une zone de texte visant à guider l'utilisateur sur la manière de lancer la procédure de piquage choisie. 
\item Le deuxième sous-panel se situe au milieu de panel2 et porte un nom différent suivant l'algorithme choisi : il va se nommer \textbf{iterationPanel} pour Dilate Difference, \textbf{sigmaPanel} pour Difference of Gaussian et \textbf{radiusPanel} pour Image Correlation. Ces sous-panels contiennent deux ou trois TextFields (zones de texte à une seule ligne) pour que l'utilisateur puisse entrer les valeurs des paramètres nécessaires au bon fonctionnement des algorithmes. 
\item Le troisième sous-panel (\textbf{widthNoisePanel}) quand à lui contient deux TextFields pour les paramètres \textbf{Square width} et \textbf{Noise Tolerance} sur lesquels nous reviendrons par la suite. 
\item Le dernier sous-panel (\textbf{debugCropPanel}) contient deux checkboxs pour les modes de debug et crop (voir plus loin dans le rapport). 
\end{itemize}
\item Le panel en bas de la fenêtre (\textbf{panel3}) contient quatre boutons qui permettent à l'utilisateur de faire fonctionner le plugin. 
\item Le panel principal (\textbf{mainPanel}) contient tous les panels cités précédemment. Sa taille détermine celle de la fenêtre du plugin. 
\end{itemize}

\begin{figure}[!h] 
\includegraphics[width=1\textwidth]{plugin3-1.png}
\caption{Organisation des panels pour l'algorithme Difference of Gaussian}
\label{panels}
\end{figure}

\subsection{Les boutons}

Les boutons doivent répondre aux clics de la souris et lancer une série d'actions correspondantes en fonction de l'algorithme de piquage choisi par l'utilisateur :

\begin{itemize}
\item Le bouton \textbf{Preview} permet de tester l'algorithme sélectionné sur une seule image du stack.
\item Le bouton \textbf{Apply} permet d'appliquer l'algorithme sur l'ensemble du stack, lorsque l'utilisateur a défini les paramètres qu'il souhaite appliquer.
\item Le bouton \textbf{Help \& Info} est une aide d'utilisation du plugin.
\item Le bouton \textbf{Save Results} permet de sauvegarder le tableau de coordonnées des particules sélectionnées.% avec l'aide d'\imj.  
\end{itemize}
Pour que le plugin puisse fonctionner, l'utilisateur devra au préalable avoir ouvert un stack d'image à l'aide d'\imj.

\subsection{Algorithmes et paramètres d'entrée}

Lorsque l'utilisateur sélectionne un algorithme dans le menu déroulant, il doit ensuite choisir plusieurs paramètres.
Certains sont communs à tous. Il s'agit de la \textbf{Noise Tolerance} et de la largeur du carré que l'utilisateur souhaite utiliser pour faire le stack des particules sélectionnées (\textbf{Square width}) s'il décide de se servir de la fonction \texttt{crop} du plugin.

\subsubsection{Image Correlation}

Le principe de cette méthode est de comparer une image contenant un cercle avec l'image sur laquelle on veut sélectionner les particules. On obtient alors une nouvelle image sur laquelle on peut voir des cercles correspondant aux particules de l'image de base qui ont à peu près le même diamètre que le cercle dessiné précédemment. Ici, on fait varier la taille du cercle afin de sélectionner des particules de différentes tailles.\\
\noindent
Pour cet algorithme de piquage, l'utilisateur doit entrer le rayon minimal/maximal (\textbf{radius min, radius max}) des particules à sélectionner, ainsi que la valeur de l'incrémentation (\textbf{radius inc}). Pour un résultat optimal, avant de lancer la sélection des particules avec cet algorithme, l'utilisateur devra traiter l'image pour éliminer un maximum de bruit de fond (utilisation de filtres, modification du contraste/luminosité, sélection des contours, application d'un threshold).

\subsubsection{Difference of Gaussian}

La différence de Gauss est une technique qui consiste en la soustraction d'une version floutée de l'image d'origine à une autre version moins floutée de cette même image.\\
Ici, l'utilisateur doit entrer les valeurs de \textbf{sigma1} et \textbf{sigma2} qui vont être utilisées pour appliquer les filtres gaussiens.

\subsubsection{Dilate Difference}

Cette méthode repose sur le même principe que la différence de Gauss à ceci près que l'image d'origine n'est pas floutée mais on lui applique un certain nombre de cycles de dilatation, on obtient alors des particules grossies.\\
Dans le cas de la sélection de cet algorithme, l'utilisateur devra entrer les nombres de cycles de dilatation qu'il souhaite appliquer(\textbf{iteration1, iteration2}).

\section{Paramètres de sortie}

Lorsque l'utilisateur a choisi un algorithme et l'a appliqué au stack, le plugin renvoie (si l'utilisateur le demande) un tableau contenant les coordonnées (\emph{x} et \emph{y}) des particules sélectionnées, ainsi que leur position dans le stack (\emph{slice}). Il pourra par la suite le sauvegarder gr\^ace au menu d'\imj. \\
De plus, si l'utilisateur en fait la demande, un stack contenant autant d'images que de particules sélectionnées est créé.

\section{Les classes}

Le programme est divisé en seize classes distinctes:
\begin{list}{•}
\item Une première partie regroupe six classes qui permettent de gérer l'interface graphique.
\item
\item Une seconde partie regroupe les algorithmes de piquage, elle est composée de quatre classes.
\item La classe \texttt{AlgoFactory} sert de pivot au programme en fonction du choix d'algorithme de l'utilisateur (adaptation de l'interface et algorithme de piquage).
\item La classe \texttt{Attributes} est un singleton (instanciable qu'une seule et unique fois) et contient tous les paramètres à entrer par l'utilisateur pour faire fonctionner les algorithmes.
\item La classe \texttt{Cropper} est une fonction subsidiaire du piquage permettant de créer un stack dans lequel chaque image contient une particule piquée (si l'utilisateur le demande).
\item La classe \texttt{FFTMath} contient la méthode permettant de faire la Corrélation d'images.
\item Les classes \texttt{About} et \texttt{InfoHelp} contiennent des informations et une aide sur l'utilisation du plugin.
\end{list}

\subsection{Les classes de l'interfaces}

La classe \texttt{Pick\_EM} fait le lien entre \imj ~et la classe \texttt{PickFrame} qui est une JFrame, elle permet de créer l'interface graphique.\\
Vient ensuite la classe \texttt{PickPanel} qui permet d'adapter l'interface en fonction de l'algorithme. En découlent \texttt{PanelDilateDiff}, \texttt{PanelDoG} et \texttt{PanelImCorr} pour leurs algorithmes respectifs (Dilate Difference, Difference Of Gaussian et Image Correlation).

\subsection{Les classes des algorithmes}

La classe \texttt{Picker} est utilisée pour appeler les algorithmes. Les algorithmes étant \texttt{DialteDiff}, \texttt{DoG} et \texttt{ImCorr}.
Pour faire fonctionner ces algorithmes, la classe \texttt{Attributes} renvoie les valeurs des paramètres entrés par l'utilisateur.


\newsavebox{\fmbox}
\newenvironment{fmpage}[1]
     {\begin{lrbox}{\fmbox}\begin{minipage}{#1}}
     {\end{minipage}\end{lrbox}\fbox{\usebox{\fmbox}}}

\chapter{Réalisation}

Sous \imj, il existe deux types de plugins : \emph{PlugInFilter} et \emph{PlugInFrame}. Étant donné que nous avions besoin de différents objets graphiques, nous avons choisi d'utiliser \emph{PlugInFrame}. C'est la classe \texttt{Pick\_EM} qui en hérite et permet de lancer notre plugin à son appel par l'intermédiaire de la barre de menus d'\imj.

\section{Interface graphique}

\subsection{Panneaux et boutons}

Nos panneaux dérivent de la classe \emph{JPanel}, elle-même issue de la classe \emph{Panel}. Cette dernière fournit un composant Container permettant d'accueillir d'autres composants graphiques (sous-panneaux).\\

Le premier panneau (\textbf{panel1}) contient une zone de texte (\emph{JLabel}) afin d'afficher un message d'aide, ainsi qu'un menu déroulant (\emph{JComboBox}) pour le choix des algorithmes. \\

Le panneau central (\textbf{panel2}) est vide au lancement du plugin et son contenu varie en fonction de l'algorithme sélectionné. Par exemple pour l'algorithme Difference of Gaussian, les sous-panneaux sont créés dans la classe \texttt{panelDoG}. Pour cet algorithme, le \textbf{panel2} comprend :
\begin{itemize}
\item \textbf{infoPanel} contenant un \emph{JLabel} indiquant le type d'image requis.
\item \textbf{sigmaPanel} et \textbf{widthNoisePanel} qui contiennent des \emph{JLabel} et \emph{JTextField} afin de créer une zone dans laquelle peut entrer les paramètres nécessaires au déroulement du piquage. 
\item \textbf{debugCropPanel} quand à lui contient deux cases à cocher (\emph{JCheckBox}) pour activer ou non les modes de débogage et de crop. 
\end{itemize}
Les classes \texttt{panelImCorr} et \texttt{panelDilateDiff} servent à la création des sous panneaux des algorithmes Image Correlation et Dilate Difference respectivement. \\

Le dernier panneau (\textbf{panel3}) comporte quatre boutons (\emph{JButton}) devant répondre aux clics de la souris à l'aide d'un \emph{ActionListener}. \\

Enfin, le panneau principal (\textbf{mainPanel}) contient tous les panels cités précédemment. Sa taille détermine celle de la fenêtre du plugin. \\
Ces quatre panneaux sont crées dans la classe \texttt{PickFrame}, qui hérite de la classe \texttt{JFrame}. \texttt{PickFrame} peut également accéder à des méthodes de la classe \texttt{ActionListener}. \\

La figure suivante (Figure \ref{panneauxDetail}) donne un aperçu plus visuel de l'organisation de ces différents panneaux.
\begin{figure}[!ht] 
\begin{center}
\includegraphics[width=0.8\textwidth]{plugin3-1.png}
\caption{Organisation des panels pour l'algorithme Difference of Gaussian}
\label{panneauxDetail}
\end{center}
\end{figure}
\pagebreak

La classe \texttt{PickFrame} permet donc de générer l'interface graphique du plugin. De plus, elle permet de réagir lors d'un clic sur un bouton. C'est la méthode \texttt{actionPerformed()} qui vérifie le nom de l'algorithme choisi par l'utilisateur parmi ceux proposés dans la JComboBox et permet d'afficher le panel2 qu'il faut. Ci-après un extrait de la méthode \texttt{actionPerformed()} :

\begin{center}
\begin{fmpage}{15cm}
\begin{small}
\begin{lstlisting}[breaklines=true, breakatwhitespace=true]
public void actionPerformed(ActionEvent e) {
  String command = e.getActionCommand();
  String comboSelection = null;
  if (command.equals("comboBoxChanged")){
    JComboBox cb = (JComboBox)e.getSource();
    comboSelection = (String) cb.getSelectedItem();
    panel2.removeAll();
    // Allows the panel2's update
    mainPanel.remove(panel1);
    mainPanel.remove(panel2);
    mainPanel.remove(panel3);
    panel2 = AlgoFactory.algorithm.getPickPanel(comboSelection);
    mainPanel.add(panel1);
    mainPanel.add(panel2);
    mainPanel.add(panel3);
    mainPanel.repaint();
    pack();
  }
\end{lstlisting}
\end{small}
\end{fmpage}
\end{center}

Comme nous avions quelques soucis d'actualisation des panneaux, lors de l'implémentation, nous avons décidé de les supprimer puis de les recréer lors de chaque changement d'algorithme. Le chargement du bon panel2 se fait gr\^ace à \texttt{getPickPanel} de la classe \texttt{AlgoFactory}. \texttt{pack()} permet de recadrer la fenêtre en fonction de son contenu. 

\begin{center}
\begin{fmpage}{15cm}
\begin{small}
\begin{lstlisting}
if (command.equals("Apply")){
  comboSelection = (String)algoList.getSelectedItem();
  Attributes.getInstance();
  coordXYZ = AlgoFactory.algorithm.getPicker(comboSelection);
  IJ.showStatus("End of picking");
}
else if (command.equals("Preview")){
  comboSelection = (String)algoList.getSelectedItem();
  Attributes.getInstance();
  AlgoFactory.algorithm.getPickerPreview(comboSelection);
  IJ.showStatus("End of Preview");
}
else if (command.equals("Show Results")){
  ToCSV.generateCsvFile(coordXYZ);
}
\end{lstlisting}
\end{small}	
\end{fmpage}
\end{center}

\paragraph*{}
Grâce à cette série structures conditionnelles, il est possible d'appliquer les méthodes de piquage appropriées en fonction du choix de l'utilisateur. Le lancement de la procédure ne se fait que si ce dernier clique sur les boutons de prévisualisation ou d'application. La tableau de résultats ne sera généré que s'il appuie sur le bouton d'affichage des résultats. 

\subsection{Affichage des résultats}

Lorsque l'utilisateur a coché la case "crop" avant de lancer la procédure de piquage, la classe \texttt{PickFrame} fait appel à la classe \texttt{Cropper} permettant de créer un stack. Les paramètres d'entrée de \texttt{Cropper()} sont une \emph{ImagePlus} (image courante du stack) et un tableau de doubles contenant les coordonnées des particules sélectionnées. \\
Lors de nos phases de tests, nous avons ajouté une autre fonction Cropper(), sans paramètres d'entrée. Nous y avons créé un tableau de coordonnées manuellement pour faciliter les essais. \\
A partir du tableau de doubles cité, la méthode \texttt{crop()} de \texttt{Cropper} fait appel à la méthode \texttt{setRoi()} d'\imj ~afin de retenir une zone carrée autour de la sélection. Cette zone va ensuite être dupliquée et ajoutée au stack sous la forme d'un \emph{ImageProcessor}. Ci-dessous un extrait de la méthode \texttt{crop()} : 

\begin{center}
\begin{fmpage}{16cm}
\begin{small}
\begin{lstlisting}
if (z == currentSlice) {
  imp.setRoi(x, y, widthCrop, widthCrop);  
  // widthCrop = taille du carre de selection entree par l'utilisateur
  img2 = new Duplicator().run(imp);
  ImageProcessor ip2 = img2.getProcessor();
  ImageProcessor impTemp = ip2.resize(widthCrop,widthCrop);
  ims.addSlice(impTemp);
}
\end{lstlisting}
\end{small}	
\end{fmpage}
\end{center}

\paragraph*{}
Cette partie du code ne va s'exécuter que si le cadre de sélection de la particule ne dépasse pas le cadre de l'image de base. Les particules dépassant n'apparaitront pas dans le stack d'imagettes, mais on pourra retrouver leurs coordonnées dans le tableau de résultats final. \\

Par ailleurs, les résultats de la sélection peuvent être affichés sous la forme d'une \texttt{ResultsTable} si on clique sur le bouton "Show Results". Elle est construite grâce au tableau de doubles cité précédemment, qui est le paramètre d'entrée de la fonction \texttt{generate-} \texttt{CsvFile()}. 

\section{Récupération des paramètres}

Le singleton de la classe \texttt{Attributes} contient une table de hashage (\emph{HashTable}) dans laquelle sont stockés tous les paramètres entrés par l'utilisateur. Ces derniers sont accessibles grâce à des clés et donc réutilisables dans les algorithmes. La méthode  \texttt{synchronized()} dans la fonction \texttt{getInstance()} (voir ci-dessous) empêche toute instanciation multiple :

\begin{center}
\begin{fmpage}{11cm}
\begin{small}
\begin{lstlisting}
public final static Attributes getInstance() {
  if (Attributes.instance == null) {
    synchronized(Attributes.class) {
      if (Attributes.instance == null) {
        Attributes.instance = new Attributes();
      }
    }
  }
  return Attributes.instance;
}
\end{lstlisting}
\end{small}	
\end{fmpage}
\end{center}

\pagebreak

\section{Algorithmes}

\subsection{Comparaison langage Macro \imj ~et \java}

Nous avons commencé par implémenter les algorithmes grâce à l'outil Macro d'\imj lors de nos  phases de test. Nous les avons ensuite traduits en \java et liés à la partie interface graphique du code. Vous trouverez ci-après un exemple de cette transformation. \\

Extrait de la Macro de la Différence de dilatation :
\begin{center}
\begin{fmpage}{16cm}
\begin{small}
\begin{lstlisting}
run("Blobs (25K)");
run("Duplicate...", "title=blobs-1.gif");
run("Duplicate...", "title=blobs-2.gif");
run("Dilate");
run("Make Binary");
selectWindow("blobs-1.gif");
run("Make Binary");
run("Dilate");
run("Options...", "iterations=2 count=1 edm=Overwrite do=Nothing");
selectWindow("blobs-2.gif");
run("Dilate");
imageCalculator("Subtract create", "blobs-2.gif","blobs-1.gif");
selectWindow("Result of blobs-2.gif");
run("Find Maxima...", "noise=3 output=[Point Selection]");
\end{lstlisting}
\end{small}	
\end{fmpage}
\end{center}

\paragraph*{}
Équivalent en langage \java :
\begin{center}
\begin{fmpage}{16cm}
\begin{small}
\begin{lstlisting}
ImagePlus imp = WindowManager.getCurrentImage();
ImagePlus imp1=new Duplicator().run(imp);
ImagePlus imp2= new Duplicator().run(imp1);
imp1.setSlice(currentslice);
imp2.setSlice(currentslice);
IJ.run(imp1, "Make Binary", "calculate");
IJ.run(imp2, "Make Binary", "calculate");
IJ.run(imp1, "Options...", it1);
IJ.run(imp1, "Dilate", "slice");
IJ.run(imp2, "Options...", it2);
IJ.run(imp2, "Dilate", "slice");
ic = new ImageCalculator();
ImagePlus imp3 = ic.run("Subtract create", imp2, imp1);
ImageProcessor ip3 = imp3.getProcessor();
ip3.invert();
Polygon points = mf.getMaxima(ip3, tolerance, excludeOnEdges);
\end{lstlisting}
\end{small}	
\end{fmpage}
\end{center}

\subsection{Méthodes de piquage}

Lorsque l'utilisateur fait le choix d'un algorithme de piquage parmi ceux qui lui sont proposés, cela fait appel à la classe \texttt{AlgoFactory} contenant plusieurs méthodes switch :
\begin{itemize}
\item \texttt{getPickPanel()} permet de récupérer le nom de l'algorithme choisi et d'afficher le panel2 correspondant.
\item \texttt{getPicker()} permet de lancer le piquage lorsque l'utilisateur appuie sur le bouton Apply.
\item \texttt{getPickerPreview()} permet de lancer le piquage lorsque l'utilisateur appuie sur le bouton Preview.
\end{itemize}

La présence d'un constructeur privé dans cette classe supprime le constructeur public par défaut. De plus, seul le singleton peut s'instancier lui même. \\

Une fois le panel2 chargé, l'appel aux procédures de piquage ne peut se faire que si l'on clique sur les boutons de prévisualisation (Preview) ou d'application à l'ensemble du stack (Apply). \\
Les paramètres entrés par l'utilisateur sont sauvegardés dans la table de hashage grâce à la fonction \texttt{getInstance()} de la classe \texttt{Attributes}. Dans le cas où l'utilisateur entrerait des valeurs non numériques, c'est \imj ~qui se chargerait de gérer les erreurs. \\
Les paramètres sont ensuite récupérés, pour l'algorithme, par la fonction \texttt{setAttributes()} de la classe \texttt{PanelDoG} (pour suivre notre scénario). \\
 L'algorithme est par la suite appelé par la méthode \texttt{sliceSelection()} (Apply) ou par \texttt{picking()} (Preview). \\

La méthode \texttt{picking()} donne l'image courante du stack en paramètre de la méthode \texttt{pick()} alors que \texttt{sliceSelection()} parcourt le stack et appelle \texttt{pick()} autant de fois qu'il y a d'images dans le stack. \\

La méthode \texttt{pick()} quand à elle permet de lancer l'algorithme sur la sélection. Elle prend une ImagePlus et le numéro de l'image dans le stack en paramètres. Cette fonction récupère les paramètres entrés par l'utilisateur (grâce à \texttt{hashAttributes.get()}) et renvoie un tableau de résultats (X, Y, Slice). Voici un extrait du code de l'algorithme Difference of Gaussian :

\begin{center}
\begin{fmpage}{16cm}
\begin{small}
\begin{lstlisting}
Hashtable<String, String> hashAttributes = Attributes.getAttributes();
String sigma1 = hashAttributes.get("sig1");

imp.setSlice(currentslice);
ImagePlus imp1 = new Duplicator().run(imp);

String si1 = "sigma=" + sigma1;

imp1.setSlice(currentslice);
IJ.run(imp1, "Gaussian Blur...", si1);
\end{lstlisting}
\end{small}
\end{fmpage}
\end{center}

Ici, nous obtenons le paramètre \textbf{sigma1} grâce à la table de hashage et la clé "\texttt{sig1}". Nous avons besoin de ce dernier pour appliquer le filtre gaussien sur l'image courante, c'est pourquoi nous le castons sous la forme d'un \emph{String}. Notons que la fonction \texttt{IJ.run()} permet de lancer une procédure \imj. \\

Nous avons choisi d'utiliser des vecteurs pour stocker les coordonnées ainsi que les numéros de slices car il nous est impossible de connaître à l'avance le nombre de particules qui vont être sélectionnées. \\
Nous avons fait en sorte de vider les tableaux de résultats entre le mode de prévisualisation et d'application, mais aussi entre deux applications ou deux prévisualisations. Ceci évite que les résultats ne s'ajoutent, ce qui fausserait la résultante du piquage. Il en est de même pour tous les algorithmes. \\

De plus, les classes contenant les algorithmes de piquage héritent de la classe \texttt{Picker}. Celle-ci contient la méthode \texttt{resultConverter()} permettant regrouper les différentes coordonnées ainsi que les numéros des slices dans un seul et même tableau. De plus, tous les attributs communs aux algorithmes sont déclarés dans cette classe. 

\section{Autres classes}

La classe \texttt{FFTMath} est issue de la classe \texttt{FFTMath} d'\imj, que nous avons modifié afin de pouvoir réaliser la corrélation d'images dans l'algorithme Image Correlation. Nous avons choisi de le modifier afin d'éviter un affichage graphique intempestif, qui avait pour effet de ralentir le déroulement du programme. De plus, cela n'avait pas n'intérêt particulier pour l'utilisateur. \\

La classe \texttt{About} permet d'afficher les auteurs du plugin et le moyen de nous contacter si besoin. \\

La classe \texttt{InfoHelp} affiche une aide sur le fonctionnement du plugin si l'utilisateur  clique sur le bouton "Help \& Info". 

\section{Applications}

Le résultat affiché sur l'image correspond aux positions des particules sur l'image courante, ou la dernière image dans le cas d'un stack.
L'utilisateur peut choisir d'afficher un tableau contenant les coordonnées (abscisses, ordonnées et positions dans le stack) des particules sélectionnées et pourra le sauvegarder. \\

De plus, s'il le désire, un stack contenant les particules sélectionnées aux positions obtenues est créé (éliminant les particules trop près du bord de l'image) et affiché.

\begin{figure}[!ht]
\begin{center}
 \begin{minipage}{.450\linewidth}
  \includegraphics[width=0.75\textwidth]{cropblob.png}  
 % \caption{Difference de Gaussienne (blobs)}
 \end{minipage} \hfill
\begin{minipage}{.450\linewidth}
  \includegraphics[width=0.5\textwidth]{cropprotDog.png}   
  %\caption{Difference de Gaussienne (protéines)}
 \end{minipage} \hfill
\caption{Exemple d'images formant le stack de particules (blobs et protéines)}
\end{center}
\end{figure}

\subsubsection*{Statistiques}

\begin{itemize}
\item[•] La Différence gaussienne permet d'obtenir les résultats suivants :
\end{itemize}

\begin{figure}[!ht]
\begin{center}
 \begin{minipage}{.450\linewidth}
  \includegraphics[width=0.75\textwidth]{blobsDog.png}  
 % \caption{Difference de Gaussienne (blobs)}
 \end{minipage} \hfill
\begin{minipage}{.450\linewidth}
  \includegraphics[width=1\textwidth]{protDog.png}   
  %\caption{Difference de Gaussienne (protéines)}
 \end{minipage} \hfill
\caption{Différence Gaussienne (blobs et protéines)}
\end{center}
\end{figure}

Nous constatons que ce piquage est efficace sur notre image de protéines membranaires. Nous l'avons également testé sur une image de blobs, qui est une image de référence pour les essais sur \imj. Cette technique semble tout aussi bien marcher sur les blobs. \\

\begin{itemize}
\item[•] La Différence de dilatation permet d'obtenir les résultats suivants :
\end{itemize}

\begin{figure}[!ht]
\begin{center}
 \begin{minipage}{.450\linewidth}
  \includegraphics[width=0.75\textwidth]{blobDilate.png}  
 % \caption{Difference de Gaussienne (blobs)}
 \end{minipage} \hfill
\begin{minipage}{.450\linewidth}
  \includegraphics[width=1\textwidth]{protDilate.png}   
  %\caption{Difference de Gaussienne (protéines)}
 \end{minipage} \hfill
\caption{Différence de dilatation (blobs et protéines)}
\end{center}
\end{figure}

Nous observons que cet algorithme semble bien fonctionner sur les blobs, alors que sur les protéines il y a beaucoup trop de points de piquage. Même en modifiant le paramètre de la "Noise Tolerance" nous obtenons des résultats de piquage similaires. \\

\begin{itemize}
\item[•] La Corrélation d'images ne permet pas d'obtenir de résultats de piquage exploitables avec ces deux types d'images. Nous avons également testé cet algorithme avec des images de particules virales. Nous obtenions de bon résultats, cependant, pour des raisons de confidentialité, nous ne pouvons pas intégrer ces images dans notre rapport. \\
\end{itemize}

\begin{itemize}
\item[•] Résultats :
\end{itemize}

Nous avons réalisé une série de tests statistiques afin de comparer l'efficacité de nos algorithmes sur les différentes images. Les résultats obtenus sont regroupés dans le tableau (Table \ref{tableau}) suivant :

\begin{table}[h]
\begin{center}
\begin{tabular}{|c|c|c|c|c|}
\hline
\textbf{Images} & \textbf{Variables} & \textbf{DoG} & \textbf{Dilate Difference} & \textbf{Image Correlation} \\
\hline
Blobs & Vrai Positif (VP) & 63 & 61 & null \\
	& Vrai Négatif (VN) & 3 & 0 & null \\
	& Faux Positif (FP) & 9 & 2 & null \\
	& Faux Négatif (FN) & 0 & 1 & null \\
	& Sensibilité (SE) & 1 & 0.98 & null \\
	& Spécificité (SP) & 0.25 & 0 & null \\
\hline
Protéines & Vrai Positif (VP) & 167 & infini & null \\
	& Vrai Négatif (VN) & 8 & infini & null \\
	& Faux Positif (FP) & 16 & infini & null \\
	& Faux Négatif (FN) & 8 & infini & null \\
	& Sensibilité (SE) & 0.95 & null & null \\
	& Spécificité (SP) & 0.5 & null & null \\
	\hline
\end{tabular}
\end{center}
\caption{Tableau de statistiques d'efficacité des algorithmes de piquage}
\label{tableau}
\end{table}

Les \textbf{Vrais Positifs} sont les particules devant être sélectionnées et qui le sont par l'algorithme. Les \textbf{Vrais Négatifs} représentent tout ce qui ne doit pas être sélectionné et qui ne l'est pas. Les \textbf{Faux Positifs} correspondent à tout ce qui ne doit pas être sélectionné mais qui l'est. Enfin, les \textbf{Faux Négatifs} sont les particules devant être sélectionnées mais qui ne le sont pas. \\ 
Le \textbf{Sensibilité}~\cite{stats:url} est la probabilité qu'une particule devant être piquée le soit. Une mesure de la sensibilité s'accompagne toujours d'une mesure de la spécificité. La \textbf{Spécificité} est la probabilité de ne pas sélectionner ce qui ne doit pas l'être. La sensibilité et la Spécificité sont obtenues par les formules suivantes : \\

\begin{center}
SE = $\frac{\text{VP}}{\text{VP+FN}}$ et SP = $\frac{\text{VN}}{\text{VN+FP}}$ \\
\end{center}

D'après la Table \ref{tableau}, nous remarquons que c'est l'algorithme de Différence gaussienne qui a la meilleure spécificité et sensibilité pour les blobs. En effet, il y a plus de particules sélectionnées avec ce dernier qu'avec la Différence de dilatation, alors que cet algorithme a été créé pour les blobs au départ. Cependant, nous constatons qu'il y a plus de faux positifs avec la Différence gaussienne. \\
En ce qui concerne les protéines, la Différence de dilatation est inefficace car elle ne fait pas de distinction entre le bruit et les particules. La Différence gaussienne, quant à elle, est assez efficace en ce qui concerne la sélection de ces dernières. 

\section{Ajout d'un nouvel algorithme au plugin}

Voici la marche à suivre pour ajouter un nouvel algorithme de piquage à notre plugin :

\begin{itemize}
\item Ajouter le nom de l'algorithme dans la liste des noms proposés par la \emph{JComboBox} dans la classe \texttt{PickFrame}.
\item Ajouter un nouveau "\textit{case}" dans chaque "\textit{switch}" de la classe \texttt{AlgoFactory}.
\item Créer une nouvelle classe \texttt{PanelAlgorithme} dans laquelle doivent se trouver :
	\begin{itemize}
	\item un \emph{JLabel} pour la phrase d'indication d'utilisation de l'algorithme,
	\item tous les \emph{JLabel}, \emph{JCheckBox}, \emph{JButton}, \emph{JTextField}, \emph{JPanel} nécessaires pour le fonctionnement de l'algorithme,
	\item une méthode \texttt{setAttributes()} pour récupérer les paramètres entrés par l'utilisateur,
	\end{itemize}
\item Créer une nouvelle classe \texttt{Algorithme} pour la procédure de piquage suivant le modèle de ceux déjà implantés. 
\end{itemize}






















\chapter*{Conclusion}
\addcontentsline{toc}{chapter}{Conclusion}

\chapter*{Bibliographie}
\addcontentsline{toc}{chapter}{Bibliographie}

\chapter*{Annexes}
\addcontentsline{toc}{chapter}{Annexes}
\bibliographystyle{is-unsrt}
\bibliography{rapport}



\end{document}