\documentclass[11pt]{beamer}
\usepackage[OT1]{fontenc}
\usepackage[utf8x]{inputenc}
\usepackage[frenchb]{babel}
\usetheme{PickEM}
\title{Développement d'un utilitaire de sélection de particules observées au microscope électronique}
\subtitle{Pick\_EM}
\author{\textsc{Faux} - \textsc{Héricé} - \textsc{Paysan-Lafosse} - \textsc{Sansen} %\\ Sous la direction de J-C \textsc{Taveau}
}
\institute[Universite Bordeaux 1 \& 2] % (optionnel, mais nécessaire la plupart du temps)
 {
 Master 1 Bioinformatique \\ Projet de programmation sous la direction de Jean-Christophe \textsc{Taveau} \\ 
 \begin{figure}[h]
  \begin{center}
  \includegraphics[width=0.25\columnwidth]{logounibdx.png}
  \hspace{1cm}
  \includegraphics[width=0.3\columnwidth]{banniere_cbmn.png}
  \end{center}
 \end{figure}
 }
%\logo{%
 %   \includegraphics[width=2cm,height=2cm,keepaspectratio]{logounibdx.png}~%
  %  \includegraphics[width=2cm,height=2cm,keepaspectratio]{banniere_cbmn.png}~%
   
%}
%
% \begin{figure}[h]
%  \begin{center}
%  \includegraphics[width=0.15\columnwidth]{logounibdx.png}
%  \includegraphics[width=0.2\columnwidth]{banniere_cbmn.png}
%  \end{center}
% \end{figure}
\begin{document}

\frame{\titlepage}
\section*{Sommaire}
\begin{frame}
  \tableofcontents
\end{frame}      
\section{Introduction}
\begin{frame}
\frametitle{Introduction}
\begin{block}{CBMN}
Laboratoire de Chimie et Biologie des Membranes et Nanoobjets 
\end{block}
\begin{block}{ACMPC}
l'équipe Architectures des Complexes Membranaires et Processus Cellulaires
\end{block}
\end{frame}

\subsection{Contexte}
\begin{frame}
\frametitle{Contexte}
\begin{block}{}
\begin{itemize}
\item Piquage de particules
\item Récupération de coordonnées
\end{itemize}
\end{block}
\end{frame}

\subsection{Objectif}
\begin{frame}
\frametitle{Objectif}
\begin{block}{}
ici on ecrit les objectifs
\end{block}
\end{frame}

\section{Analyse}
\begin{frame}
\frametitle{Analyse}
\begin{block}{}
ici on ecrit l'analyse
\end{block}
\end{frame}

\subsection{Besoins Fonctionnels}
\begin{frame}
\frametitle{Objectif}
\begin{block}{}
ici on ecrit les besoins fonctionnels
\end{block}
\end{frame}

\subsection{Besoins non Fonctionnels}
\begin{frame}
\frametitle{Objectif}
\begin{block}{}
ici on ecrit les besoins non fonctionnels
\end{block}
\end{frame}

\section{Conception}
\begin{frame}
\frametitle{Conception}
\end{frame}

\section{Réalisation}
\begin{frame}
\frametitle{Réalisation}
\end{frame}


\section{Conclusion}
\begin{frame}
\frametitle{Conclusion}
\end{frame}

\begin{frame}
\frametitle{Remerciements}

\begin{block}{Merci!!! ;-)}
merci beaucoup pour votre attention!
\end{block}
\end{frame}

\end{document}